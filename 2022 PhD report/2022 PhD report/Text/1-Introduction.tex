\section{Introduction}
% Delete the text and write your Introduction here:
%------------------------------------

Atmosphere models and interior models are often built separately and rely on two different underlying physical starting points. For the former, that atmospheres are in radiative convective equilibrium, and for the latter, that interiors are fully convective. However atmospheric models will, in most cases, give a convective profile for the deep atmosphere. Hence we can link these two separate models based on the premise that in such a region the physics are the same. The linkage of two separate models requires continuity of the four following quantities, temperature, pressure, gravity and mean molecular mass. This in itself leads to more of a mathematical and numerical problem then a physical one, should we start from the premise that there is indeed a space of parameters that leads to a solution. For the subsequent work we use the Exo-REM 1D radiatif convectif atmosphere model from the LESIA and the Exoris interior model from the Observatoire de la côte d'Azur. Atmosphere models come in many shapes and sizes but to accurately describe an atmosphere they need to posses a chemistry model preferably out of equilibrium, a cloud model and an parameterizable interior and exterior radiative flux. Exo-REM has the advantage of all of these aspects. However we shall not use the cloud model for the work presented here. Again interior models are plentiful, but fundamentally they need to use the best possible state equations of the elements that we might find inside a planet. Exoris contains such equations and one can change them as better ones come available. \par
One could wonder, what is the point in linking two distinct models? The answer to this question can be found in the internal radiative flux or conversly the internal temperature. It is hard to understand how one can fix this value in atmosphere models. For a gas planet of our solar system we can deduce it quite easily as we know the effective temperature and we know the stellar flux. However for a given gaseous exoplanet it is harder, especially so for irradiated planets where the effective temperature is mainly given by the stellar irradiation. Yet this internal temperature plays an important role in the atmospheres dynamics especially for young gaseous planets. Furthermore as a planet ages it evacuates internal energy, as such the internal temperature, in a sense, is a date stamp. As such we see the obvious conundrum, how do we access this internal temperature? The answer lies in the linkage of the models as stated, indeed if we can fix the thermal profile in the convective region of the atmosphere with an interior model, we can equally fix the internal temperature for a given set of planetary parameters. This will hence be the focal point of the subsequent work.

