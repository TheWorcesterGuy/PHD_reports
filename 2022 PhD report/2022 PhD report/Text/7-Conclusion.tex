\section{Conclusion and improvements}
% Delete the text and write your Conclusion here:
%------------------------------------
We see that it is possible to link the Exorem and Exoris model into an evolution model. The obtained results are highly satisfactory when compared to the literature. The application of the model on HD209458b shows that it is possible given observational constraints to estimate planetary parameters. This example also shows further work required. Interpolations to obtain the data for a given planet can only be trustworthy if the dataset is not sparsely populated in the region of interest and the values are given with an error bellow that of the variation in the simulated properties. Work will be required to take into account these uncertainties in order to derive correct error bars on model outputs. Further work is required on the Exoris side of the model, it is common practice to use water instead of Helium to emulate heavier elements, as such this needs to be rapidly implemented before any publication. Adding in cloud cover seems like the next logical step as well as populating the dataset with a range of core values and metallicity. This model goes further than various models currently published due to the out of equilibrium atmosphere chemistry, the use of a planetary core and the capacity to take into account stellar irradiation. In the medium term it seems essential to diversify interior equations of state, most notably if one wishes to derive Neptune like planetary parameters. Further more it would be interesting to compare evolution curves of cloudy models to fingering convection models, this might help differentiate between the two atmospheric dynamics.