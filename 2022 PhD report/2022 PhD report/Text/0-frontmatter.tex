\begin{frontmatter}
%
% Title:
%------------------------------------
\title{%
2022 PhD report\\
\small Planetary Model linking annd futur prospects  % A good idea is to have the subject code and name as subtitle
}
%
% Authors:
%------------------------------------
% List an author with name ' Firstname Middlename Lastname ' like this:
% F. M. Lastname
\author[Observatoire de Paris-Meudon]{C. Wilkinson} 
\author[Observatoire de Paris-Meudon]{B. Charnay}
\author[Observatory De La Côte D'azur]{S. Mazevet}
\author[Observatoire de Paris-Meudon]{AM. Lagrange}
\address[Observatoire de Paris-Meudon]{Obsrvatoire de Paris-Meudon - Lesia}
\address[Observatory De La Côte D'azur]{Observatory De La Côte D'azur}
%
% Date:
%------------------------------------
%
\newdate{dateName}{15}{07}{2022} % edit the date here, ' dateName ' has to match on these two lines.
\renewcommand*{\today}{\MonthYearDateFormat\displaydate{dateName}} 
% Options for displaying date: \MonthYearDateFormat,  \DayMonthYearDateFormat or \YearDateFormat
%
% Abstract:
%------------------------------------
\NameOfAbstract{Abstract} % Change abstract title here. If you write in Norwegian, write 'Sammendrag' (nb) or 'Samandrag' (nn)
\begin{abstract}
% Delete the text and write your abstract here:
%------------------------------------
To model the thermal evolution of Jupiter-like exoplanets and to link observational data with atmosphere and internal models, we build of two such models, Exorem and Exoris. We show that it is possible to link both atmosphere and interior models as to constrain the both and derive planetary parameters such as the internal energy flux, the radius and the spectral radiosity. We hence compute the parameters for a grid of models with varying internal and irradiation energy fluxes and masses. We show that our results are in agreement with that which is already published. We apply the combined model on the example of HD209458b where we can derive an internal temperature of 320K.\\
\\
"Numerical tools allow mediocre physicists like me to remain relevant in physics nowadays" S. Charnoz

\end{abstract}
%
\end{frontmatter}
%
%
% Table of contents:
%------------------------------------
% If the report is very long for some reason (over 4 or 5 pages), use a table of contents.
% Uncomment everything below the line ---- to get table of contents (ctrl + /) (the / on numberpad):
%-------------
%
% \ 
% \vspace{1cm}

% \begin{minipage}{\textwidth}
%     \tableofcontents
% \end{minipage}
% \clearpage